\documentclass{tufte-book}

\hypersetup{colorlinks}% uncomment this line if you prefer colored hyperlinks (e.g., for onscreen viewing)

%%
% Book metadata
\title{Cyber Peripherals\thanks{Hamish, Lumpley, Avery and the RPGtalk slackers }}
\author[Cromlyn Games]{Cromlyn Games}
\publisher{Publisher of This Book}

%%
% If they're installed, use Bergamo and Chantilly from www.fontsite.com.
% They're clones of Bembo and Gill Sans, respectively.
%\IfFileExists{bergamo.sty}{\usepackage[osf]{bergamo}}{}% Bembo
%\IfFileExists{chantill.sty}{\usepackage{chantill}}{}% Gill Sans

%\usepackage{microtype}

%%
% Just some sample text
\usepackage{lipsum}

%for fancy lists
\usepackage{tikz}
\usetikzlibrary{shadows}
\newcommand{\mylist}{\tikz[overlay]\draw(-.2,-.2)--(-.2,.4) [path fading=east](-.2,.15)--(.1,.15);} %adds the |- shape to the start of each list item
\newcommand{\mylistend}{\tikz[overlay]\draw(-.2,.15)--(-.2,.4) [path fading=east](-.2,.15)--(.1,.15);} %adds the |- shape to the start of each list item
\newcommand{\myitem}{\item[\mylist]} %defines the scope of the mylist command to be 2nd level sublists
\newcommand{\myitemend}{\item[\mylistend]} %defines the scope of the mylist command to be 2nd level sublists
%\renewcommand\labelitemi{---}  % turns bullets into long dashes

%%
% For nicely typeset tabular material
\usepackage{booktabs}

%%
% For graphics / images
\usepackage{graphicx}
\setkeys{Gin}{width=\linewidth,totalheight=\textheight,keepaspectratio}
\graphicspath{{graphics/}}

% The fancyvrb package lets us customize the formatting of verbatim
% environments.  We use a slightly smaller font.
\usepackage{fancyvrb}
\fvset{fontsize=\normalsize}

%%
% Prints argument within hanging parentheses (i.e., parentheses that take
% up no horizontal space).  Useful in tabular environments.
\newcommand{\hangp}[1]{\makebox[0pt][r]{(}#1\makebox[0pt][l]{)}}

%%
% Prints an asterisk that takes up no horizontal space.
% Useful in tabular environments.
\newcommand{\hangstar}{\makebox[0pt][l]{*}}

%%
% Prints a trailing space in a smart way.
\usepackage{xspace}

% Prints the month name (e.g., January) and the year (e.g., 2008)
\newcommand{\monthyear}{%
  \ifcase\month\or January\or February\or March\or April\or May\or June\or
  July\or August\or September\or October\or November\or
  December\fi\space\number\year
}


% Prints an epigraph and speaker in sans serif, all-caps type.
\newcommand{\openepigraph}[2]{%
  %\sffamily\fontsize{14}{16}\selectfont
  \begin{fullwidth}
  \sffamily\large
  \begin{doublespace}
  \noindent\allcaps{#1}\\% epigraph
  \noindent\allcaps{#2}% author
  \end{doublespace}
  \end{fullwidth}
}

% Inserts a blank page
\newcommand{\blankpage}{\newpage\hbox{}\thispagestyle{empty}\newpage}

\usepackage{units}

% Typesets the font size, leading, and measure in the form of 10/12x26 pc.
\newcommand{\measure}[3]{#1/#2$\times$\unit[#3]{pc}}

% Macros for typesetting the documentation

% Generates the index
\usepackage{imakeidx}
%\makeindex[name=moves, title={Index of moves}] % on reflection, not needed. 
\makeindex[name=stuff, title ={Index of elements, items}]
\makeindex % general index for playbooks and stuff

\begin{document}

% Front matter
\frontmatter

% r.1 blank page
\blankpage

% v.2 epigraphs
\newpage\thispagestyle{empty}

\openepigraph{%
Acknowledge up front that the PCs are going to win, and never sweat it. Then use the dice to escalate, escalate, escalate. We all know the PCs are going to win. What will it cost them?
}{Lumpley}
\vfill

\openepigraph{%
A designer knows that he has achieved perfection 
not when there is nothing left to add, 
but when there is nothing left to take away.
}{Antoine de Saint-Exup\'{e}ry}
\vfill
\openepigraph{%
The integrity of its span was rigorous as the modern program itself, yet around this had grown another reality, intent on its own agenda. This had occurred piecemeal, to no set plan, employing every imaginable technique and material. The result was something amorphous, startlingly organic.
}{ William Gibson, Virtual Light}
\vfill




% r.3 full title page
\maketitle


% v.4 copyright page
\newpage
\begin{fullwidth}
~\vfill
\thispagestyle{empty}
\setlength{\parindent}{0pt}
\setlength{\parskip}{\baselineskip}
Copyright \copyright\ \the\year\ \thanklessauthor

\par\smallcaps{Published by \thanklesspublisher}

\par\smallcaps{tufte-latex.github.io/tufte-latex/}

\par Licensed under the Apache License, Version 2.0 (the ``License''); you may not
use this file except in compliance with the License. You may obtain a copy
of the License at \url{http://www.apache.org/licenses/LICENSE-2.0}. Unless
required by applicable law or agreed to in writing, software distributed
under the License is distributed on an \smallcaps{``AS IS'' BASIS, WITHOUT
WARRANTIES OR CONDITIONS OF ANY KIND}, either eXpress or implied. See the
License for the specific language governing permissions and limitations
under the License.\index{license}

\par\textit{First printing, \monthyear}
\end{fullwidth}

% r.5 contents
\tableofcontents

\listoffigures

\listoftables

% r.7 dedication
\cleardoublepage
~\vfill
\begin{doublespace}
\noindent\fontsize{18}{22}\selectfont\itshape
\nohyphenation
Dedicated to those who appreciate \LaTeX{} 
and the work of \mbox{Edward R.~Tufte} 
and \mbox{Donald E.~Knuth}.

And to the horde of goblins who died defending them.
\end{doublespace}
\vfill
\vfill


% r.9 introduction
\cleardoublepage
\chapter*{Playing in Cyber\_Peripherals}
\label{ch:playing in Cyber_Peripherals}
Cyber\_Peripherals is a stand alone game of stories for 3-5 players in a cyberpunk world. Most of you will play the roles of neon slumdogs and one of you will be the Master of Ceremonies (MC for short). The game is a conversation between you all, with the MC to facilitate and fill in what is needed for the story. From time to time, the story will reach a point where the outcome is interesting and uncertain. The dice help out here, and add a sense of tension as their fall changes the fictional world and sparks new directions for the conversation.
The story will center around a community of individuals and how they live under the various pressures of a Zaibatsu dominated future. You're unlikely to be pulling off Shadowrun type missions, and if you are, you'll want to take a  look at the rpg The Sprawl, which is built for that. This rpg is designed to be compatible, but is more interested in the kind of stories that came later in the cyberpunk genre, those of Greg Egan or Tricia Sullivan or Nalo Hopkinson. 

In fine cyberpunk tradition, this book is pirated hackjob. Bits of the Sprawl, Urban Shadows and Apocalypse World have been recycled, paired up and duct taped together with Simple World. The playbooks are mostly original, but precious little else is. A favour is owed to the designers, testers and editors of those books.

\section {Why play Cyber\_Peripherals?}
Play if you want to play to find out what happens in a neon and chrome cyberpunk future.
Play if you want to create a story about everyday folk living on the peripheries.
Play if you are playing the Sprawl, and want to follow your Killer character home to see how they do the groceries.
Play if you want to struggle against the Man, in a world where 'Man' needs citation marks.
Play if you want to win sometimes, lose sometimes, and explore the future twenty minutes from now.

\section{How to use this book} \label{sec: how to use this book}
\marginnote{If you're an experienced player or MC of Powered By the Apocalypse type games, you probably want to check the MC agenda \nameref{sec: agenda}, principles \nameref{sec: principles} and Moves\nameref{sec: mcmoves}. The player basic moves are on page \nameref{ch:basic moves}. The playbooks are here \nameref{ch:playbooks}:
If you have crossed over from a Sprawl Game, you will also want to look at \nameref{sec: the sprawl}}

This first chapter will introduce the basic concepts of the game, what characters, stats and the other core mechanics are and how to use them. The next will introduce the Basic Moves, moves that all player characters have that the set the tone and pace of the game. Then we'll look at game setup, and following that the individual playbooks, so that each player has a unique archetype to play to, play against and create headaches for everyone else. Don't say "Zero does", say "I do".

\section{Stats} \label{sec: Stats}
Every character has six stats which represent their raw ability in different spheres of the world. As you look at them, you'll see areas where they might overlap. This is fine, just use what the triggered move demands for your character in that situation.

\begin{itemize}
	\item \smallcaps{Cool} to remain calm and focused in stressful situations
	\item \smallcaps{edge} to draw on your street smarts, street rep or 
	\item \smallcaps{Meat} use your physical talents without cyberware to get stuff done
	\item \smallcaps{Mind} to think your way around a problem with sheer intelligence, lateral thinking or obscure knoweldge
	\item \smallcaps{style} to handle a sitatuon with charismas, presence or raw sex appeal
	\item \smallcaps{synth} to interface seamlessly with technogogy
\end{itemize}

Each of these stats varies between -1 and +2 to indicate how well that specific character acts in that sphere. The higher the rating, the more competent you are.  These Stats are the same as in the Sprawl. If you get \smallcaps{take +1 forward} than you can also add that +1 onto the total for the next roll you make.

\section{ Core mechanics} \label{sec: core mechanics}
\subsection{Roll Dice}	\label{sec: roll dice}
You'll see the instruction to \smallcaps{roll+stat} quite a lot in the rules. This simply means roll two six sided dice and add them and the stat together. As a general rule getting ten or above (10+) is a strong hit  and getting seven to nine inclusive (7-9) is a weak hit. You get what you want, but there's complications. Getting below six (6-) is a miss. You might not get what you want, or if you do, it's only because it sets the MC up to tell a new bit of the story. That is then called the Mc making a move.

\subsection{Making moves} \label{sec: making moves}
Moves are specific set pieces in the genre fiction. They typically come with a trigger, a roll for someone to make, and a range of results. Nearly always, they are powerful things you can do as a player (or MC) to shape the fiction. You can definetly try and guide the story towards a situation where a move you want to use triggers, but the move must trigger in the fiction before dice hit the table. Some moves are called Basic Moves and can be used by any player character. Others are specific to playbooks.

An example: 
Jokero the Trader has just caught Greendog raiding his bins for unsold, out of date food. \\
Jokero: I've had three break-ins the last week and I've had it with these punks. I grab my SchorchTorch Hygenieator 3000, stand above the bin pit and aim down. "Oi, nedstain, where's the pallet of steak you stole last week? Better answer quick matey, or you'll be served up very well done to the meat wagon."\\
MC: "Are you bluffing?"\\
Jokero: "Nah. A trader lives by his word."\\
MC: "Right we're playing hardball then, roll +edge."\\
"Play Hardball" is the move for threats backed up by the threat of real violence. If Jokero was bluffing, it would trigger a different move like "Fast Talk" instead. Poor Greendog.\\

\subsection{Experience} \label{sec: experience}
Experience measures how entangled you are in the world, how much you've achieved and how it has left its mark on your character. Every time you \smallcaps{mark XP} it goes on your character sheet playbook. To stay compatible with the sprawl, each time you mark ten experience you'll choose a new advance for your character. The advances you can choose are listed in your character's playbook and include options like raising a stat or choosing a move from a different playbook.

\subsection{MC moves} \label{sec: intro MC Moves}
The MC makes moves based on triggers too. The triggers are a little different though: The MC makes a move when a player misses their move, when the fiction demands it and when the players are waiting for something to happen (around a table, that's tends to be when the conversation stalls and everyone looks at the MC). A full list of the MC moves can be found in \nameref{sec: mcmoves}. When the MC makes their move they may make it a soft move, where the players get a chance to avoid bad consequences, or a hard move, where the consequences happen as they say it. The MC move might appear to unrelated to what you failed at. Dosen't matter, as long as it moves the fiction along and is interesting.
The MC is also obligated to make hard moves as the fiction demands. Stand in the open firing a pistol at Corp Sec snipers, the fiction demands you get shot, maybe in the leg if the MC is feeling merciful or just interested who'll come and rescue your stupid arse. That said, the MC is not out to kill you.\footnote{in this case, hopefully a previous move would have been "tell the consquences and ask", where they remind you that standing in the open taking potshots at professional snipers is something your character would now is dumb. But maybe you need to buy getaway time for someone else. Man, they'll owe you a big favor.} They are out to make the garish, chrome and neon and dirt encrusted world of Cyber\_Peripherals feel real. They are out to challenge your characters so we can all see how they respond and develop. They are here to find out what happens.

\subsection{Agenda and Principles} \label{sec: intro agenda and princ}
The Agenda are the things the MC must do. They're a damn good guide for players too.
\begin{itemize}
	\item Play to find out what happens
	\item Make the world dirty, shiny, high-tech and unequal
	\item Don't waste your players time.
	\item Entangle the characters with the world.
\end{itemize}

If you are in doubt, do something that fulfills the agenda. The easiest way to do this consistently is the more detailed rules the MC must follow called the Principles:
\begin{itemize}
	\myitem chrome everything, then make it dirty
	\myitem make everything corporate. make everything reused.
	\myitem start with the vertical
	\myitem ask questions about the everyday. 

	\myitem address the characters, not the players
	\myitem ask loaded questions and incorporate the answers

	\myitem name everyone, make them rational
	\myitem treat your NPCs like disposable assets
	\myitemend build shifting, unstable communities
 \end{itemize}

\subsection{Tags}
A tag is a word which describes a significant fictional characteristic of something in the game. It's a quick, semi-formal shorthand for something interesting about an object, person, gang, vehicle ect. They can be positive or negative, or might be a mixed bag. For example, a '+flashy' jacket might a good bribe for a tasteless gambler, but the MC might call up the '+flashy' jacket as the reason you were noticed after failing a roll. A weapon with '+hair-trigger' might be justification for getting that first shot off, but might also be an MC follow-up move if you drop your gun.

The purpose of tags is to condense the description of things to the parts that might matter for the story. They also provide inspiration and invite different moves for players and the MC. An '+ex-military' gang is going to behave differently to a '+scavengers' gang, and an '+ex-military +scavengers' gang has told you a lot about the world in only two words. The suggestiveness of combinations of tags makes them especially useful to help improvisation at the table.

Some tags have explicit impacts on the game mechanics as well as fiction (for example +ap weapons ignore armour) but most impact on the fiction.

\subsection{Favours}
Favours are introduced from Urban Shadows. In the shadowy slums and favelas and encrusted fungal brackets on the great arcologies, Cred is useful, important even. Inside the community, \smallcaps{favours} are just as important. Favours between characters tie them together, often for more than just paying the favour off. Being owed favours lets you punch a little above your weight. Owing favours is a sign that others trust you to pay them back. Unless you don't, in which case remember, the street finds it's own uses for things.

\subsection{Gangs}
Gangs are introduced from Apocalypse World. The Sprawl is all about hard-bitten individuals who chase the mission, creds and their directives. Cyber\_Peripherals is all about communities, and sometimes bits of those communities work together. The gangs section of the rulebook has explicit mechanics for handling gang on gang conflict easily, but mostly gangs are about characters, the stories that unfold and the additional fictional positioning. You can literally have characters in three places at once. Handy.

\subsection{Fictional Positioning}
Some moves, tags or equipment don't have numerical effects in the game mechanics. Instead they give the characters fictional positioning, the option to undertake certain kinds of action. You can't give somebody a lift if you haven't got a vehicle. You can't shoot somebody unless you have a gun. You can't fix somebody's roof unless you have the materials, tools and a way to get up there safely. But maybe, with the right combination of equipment, cyberwear and help from someone giving you a boost, you can. No move triggered, no dice needed.

Sometimes fictional positioning is used to make the ground less steady. The trader accepted a '+blackmarket' shipment. That means the MC has a justification, almost a responsibility, to find out what that means for the story, by unleashing a couple of cybered-up custom agents. Someone's rolled a miss, and has a '+disloyal' gang. Well, One-Eyed Kratus has been complaining in fiction for a while now, perhaps it's time to challenge for gang leadership? 

You don't win a game like this, but if a player sets up their character to run a gang with the '+disloyal' tag , they might as well be saying "I want a story where a brutal leadership challenge happens.". Players and MCs both, giving players what they are asking for is a good thing for the game. Player's -  make your interests as explicit as you want. GMs- use them, incorporate them, even aim at them. Make sure you've all got the same expectations going in. 



%%
% Start the main matter (normal chapters)
\mainmatter

\chapter{Basic Moves} \label{ch:basic moves}

\subsection{Act under Pressure (cool)} \label{move: act under pressure}
When you race against the clock, act while in danger or act to avoid danger, \smallcaps{roll+cool}
\begin{itemize}
	\item On a 10+ you do it, no problem.
	\item On a 7-9 you do it but flinch. The MC will offer you a worse outcome, hard bargain, or ugly choice
	\item On a 6- you fail to do it. Ask the MC what the repercussions are.
\end{itemize}
This move is the same in the Sprawl. It's the move that covers situations where the outcome is uncertain but you haven't got a better move to make.

\subsection{Read the Street (edge)} \label{move: read the street}
When you read a charged situation \smallcaps{roll+edge}
\begin{itemize}
	\item On a 10+ ask up the three listed questions at any time during the scene
	\item On a 7-9+ ask one listed question
	\item On a 6- ask one listed question anyway, but be prepared for the worst.
	\begin{itemize}
		\myitem Where's my best escape route / way in/ way past?
		\myitem Where's the profit here?
		\myitem Which enemy is the biggest threat?
		\myitem What should I be on the lookout for
		\myitem How can I avoid trouble/ hide here?
		\myitem Who or what is in control here?
	\end{itemize}
\end{itemize}
This move is a modified mixup from the Sprawl and from Apocalpyse World.\footnote{interestingly, Urban Shadow's does not have a basic move dedicated to assessing a sitch, only people}
 It's the move you use to get information. The MC should be generous and truthful with answers. Confident players do interesting things. 

\subsection{Figure Someone Out (mind)} \label{move: figure someone out}
When you try to figure someone out \smallcaps{roll+mind}.
\begin{itemize}
	\item On a 10+ Ask two listed questions
	\item On a 7-9+ Ask two listed questions, and they get to ask you one.
	\item On a 6- Ask one listed question, and they get to ask you two.
	\begin{itemize}
		\myitem Who is pulling your strings?
		\myitem What's your beef with ...?
		\myitem What are your hoping to get from ...?
		\myitem How could I get you to ...?
		\myitem What do you worry might happen?
		\myitem How could I make you owe me a favour?
	\end{itemize}
\end{itemize}
This move is a modified version from Urban Shadows. \footnote{in parrallel to read the street, the Sprawl does not have a basic move for reading people. Presumably to keep the chromed up killers from empathy :) }
The you is directed to the character, but the owner must reply reasonably truthfully. This move can be carried out at range. In that case, the reflected questions might be clues, dna traces, search metadata ect that the player character left behind.


\subsection{Play Hardball(edge)} \label{move: play hardball}
When you get in someone's face threatening violence you intend to carry through \smallcaps{roll+edge}
\begin{itemize}
	\item On a 10+ NPCs do what you want. PCs choose: do what you want, or suffer the established consequences
	\item On a 7-9 for NPCs, the MC chooses one:
		\begin{itemize}
		\myitem they attempt to remove you as a threat, but not before suffering the established consequences.
		\myitem they do it, but they want revenge. Add them as a Threat.
		\myitem they do it, but want real payback. Owe them a \smallcaps{ Favour}.
		\end{itemize}
		PCs choose: do what you want, or suffer the established consequences. They get +1 forward to act against you.
	\item On a 6- it all goes to shit. Look at how much the MC is grinning.
\end{itemize}
This  move is a modified version from the Sprawl. The corporate clocks don't apply here but favours do. 

\subsection{Fast Talk (style)} \label{move: fast talk}
When you try to convince an NPC to do what you want with promises, lies or bluster \smallcaps {roll+style}. If you call in a favour they owe you before rolling \smallcaps{add +3 to roll}
\begin{itemize} 
	\item On a 10+ NPCs do what you want. 
	\item On a 7-9 NPCS do it but they modify the terms or you owe them a \smallcaps{ Favour}. 
	\item On a 6-  they can't. They might tell you why, or maybe you're just interrupted...
\end{itemize}
This move is a mashup from the Sprawl and Urban Shadows. If you want to persuade a PC, call in a favour or offer one to them.
When you call in a favour to get the +3, make sure you explain how you've done that in fiction, wether explicit appeals to their sense of honour or dark threats about hidden secrets.

\subsection{Mix it Up (meat)} \label{move: mix it up}
When you use violence to seize control of an objective, state that objective and \smallcaps{roll+meat}
\begin{itemize}
	\item On a 10+ you achieve your objective.
	\item On a 7-9 you achieve your objective but choose two.
		\begin{itemize}
		\myitem you leave something behind (MC to detail)
		\myitem you take harm as established by the fiction
		\myitem an ally takes harm as established by the fiction
		\myitemend something of value breaks
		\end{itemize}
	\item On a 6- you don't achieve your objective. Pray the MC offers you a hard bargain.
\end{itemize}

This move is a modified version from the Sprawl. The objective is very rarely "kill everyone". Those mad-dog types tend not to last in the Peripherals. It might be "recover the drugs packet.", "Impress and terrify the merchant.", "steal the drone", "escape the ambush." or similar. Dealing damage is more of a side effect than the aim. You can't tailor the objective to avoid the consequence of your choice on a 7-9 result. 

\subsection{Suffer Harm} \label{move: suffer harm}
When you suffer harm, (even 0-harm or s-harm) then reduce the harm by the level of your armour (if any), fill in the segments on your harm clock equal to the remaining harm and \smallcaps{roll+harm taken}. Try and get a low result.
\begin{itemize}
	\item On a 10+ choose 1:
		\begin{itemize}
		\myitem you're out of action: unconscious, trapped, incoherent or panicked.
		\myitem take the full harm of the attack, before it was reduce by armour. If you already did, take +1 harm.
		\myitem lose the use of a piece of cyberwear till you can get it repaired (you choose).
		\myitemend lose a body part (arm, leg, eye).
		\end{itemize}
	\item On 7-9 the MC will choose 1:
		\begin{itemize}
		\myitem you lose your footing
		\myitem you loose your grip on whatever you are holding
		\myitem you loose track of someone or something you're attending to
		\myitemend someone gets the drop on you
		\end{itemize}
	\item On a 6- you grit your teeth, but nothing worse happens to you. Yet.
\end{itemize}

This move is straight from the Sprawl. Note that the dice results are reversed; low is successful and high is complicated.

\section{Favour Moves} \label{sec: Favour Moves}
These a specialist form of basic move. Everyone has them, and they are focused on the additional game mechanic for cast-iron favours adopted from Urban Shadows. They won't come up as often as the basic moves, but you'll see them multiple times a session. 

\subsection{help/interfere}
When help/interfere with another PCs move in the fiction, \smallcaps{roll+favours they owe you} (to a max of 3+)
\begin{itemize}
	\item On a 10+ you can add +1 or -2 to their roll as you wish
	\item On a 7-9 you can add +1 or -2 to their roll as you wish, but you are exposed to complications of their move.
	\item On a 6- you are ineffectual, distracted or make the situation worse.
\end{itemize}
This is a original move.

\subsection{ Do someone a favour.}
If you do somebody a favour, they owe you a favour. If its a big favour (like saving their  kid from a burning building), maybe they owe you two. If you are unlikely to see them again, you're unlikely to get the favour back.

\subsection{Pay back a favour}
When you demand someone pays back a favour, remind them why you owe you and:
\begin{itemize}
	\item Make a PC:
	\begin{itemize}
		\myitem do you a favour at moderate cost.
		\myitem help/interfere as you ask
		\myitem answer a question honestly
		\myitem cancel a favour somebody owes them
		\myitemend transfer ownership of a favour somebody owes them to you.
	\end{itemize}
	\item Make an NPC
	\begin{itemize} 
		\myitem answer a question honestly about their organisation
		\myitem introduce you to a powerful member of their organisation
		\myitem give you a useful and worthy gift without cost
		\myitem give you +3 to fast talk them (see \nameref{move: fast talk})
		\myitem cancel a favour somebody owes them
		\myitemend transfer ownership of a favour somebody owes them to you.
	\end{itemize}
\end{itemize}

This is barely modified from Urban Shadows. It's a quite codified approach to commonsense bartering that humans have managed since before fire. Go by the fiction first, and if two players agree a favour between them is paid off, it's paid off. They might even want to keep it so that each owe the other a favour to use on help/interfere rolls.

\subsection{Refuse to honour a debt (style)}
When you refuse to pay back a favour you owe, \smallcaps{roll+style}
\begin{itemize}
	\item On a 10+ you weasel out of the current situation, but you still owe them a favour
	\item On a 7-9 you choose one:
		\begin{itemize}
		\myitem{You owe them an additional debt}
		\myitem{It leaks online. Take the +untrustworthy tag until you make it right}
		\myitem{Expect hassle from their organization or contacts}
		\end{itemize}
	\item On a 6- you can't avoid the noose. Honour your debt or the person owed chooses:
		\begin{itemize}
		\myitem{they pick two from the list above}
		\myitem{you loose all debts owed to you}
		\end{itemize}
\end{itemize}

This move is heavily modified from Urban Shadows. Sometimes you can't pay a favour back right now. Sometimes, the 'reasonable' thing they are asking for is more than you want to pay right now. This doesn't get you out of the favour, but it might let you put it off for a bit. And if you end up paying a bit of interest or having to work off a reputation, well, that's for future you to deal with.

\subsection{Drop someone's name (cool)}
When you drop the name of someone who owes you a favour, \smallcaps{roll+cool}
\begin{itemize}
	\item On a 10+ their name carries weight and opens an opportunity
	\item On a 7+ you get the opportunity you need, but the favour is spent.
	\item On a 6- you overstep. Erase the favour and brace youself.
\end{itemize}

This move is barely modified from Urban Shadows. It is useful anywhere you think that persons name will help you get by. You need to mention they owe you a favour though, and imply you could use that favour against the oppostion.

\section{Climax moves}
Like basic moves, these are open to all. They probably won't come up all that often but just in case...

\subsection{Acquire Agricultural Property (meat)}
When you hit 0000 on your Harm Clock, \smallcaps{roll +meat}
\begin{itemize}
	\item On a 10+ you survive until the medics arrive
	\item On a 7-9 you survive at a cost. Pick one:
		\begin{itemize}
		\myitem substandard treatment (-1 to a stat)
		\myitem cyberware damage (give one piece of cyberwear a negative tag)
		\myitem cleaned out. (loose all but one thing you own)
		\myitemend research division (+owned and change to a suitable sprawl playbook)
		\end{itemize}
	\item On a 6- you bleed out on the street. Sucks.
\end{itemize}

This is slightly modified, mostly to account for why a Corp would be interested in rebuilding slumscum like you. For the MC, any option is an opportunity to introduce complications. Who treats the character and why? Hook them with contracts, threats, implants, addictions, antidotes or a nice list of favours owed. Nothing is free.

\subsection{Go under the knife (cred)}
When you have new cyberware installed by a street doctor, \smallcaps{roll+cred spent} (max of +2)
\begin{itemize}
	\item the operation was a success
	\item The cyberwaer dosen't work as advertised. choose one
	\begin{itemize}
		\myitem +damaging: sometimes it hurts like hleel and eventually it will do permenant nerve damage
		\myitem + hardware decay: it works now, but it's just a matter of time
		\myitem + substandard: it works but not as well as it should
		\myitem +unreliable: sometimes it dosen't work.
	\end{itemize}
	\item there have been ... complications
\end{itemize}

This move is just as it is in the Sprawl. Obviously, you're going to need to find a back alley doc, ripper or bonesaw to carry out the operation. Using go under the knife to acquire cyberware is discussed in \nameref{ch:cyberware}. Fair prices for gear (including cyberwear) are listed in \nameref{ch:assets}.
There's no option for corporate sponsorship because frankly, why would they be interested in you? Come back when you're a pro with a reputation, or at least a Sprawl playbook.

\chapter{Preparing to play} \label{ch:preparing to play}

UNFINISHED. to be fleshed out in more detail.\\

Step 0: Define the corporations (or import them from your sprawl game)\\
Step 1: Choose a playbook. Unique playbooks in this case, you're the archetype of your community.\\
Step 2: Name and describe your character.\\
Step 3: Assign Stats
Step 4: Choose Cyberware
Step 5: Choose playbook Moves
Step 6: Choose gear
Step 7: Build your community

\section{install Sprawl drivers?} \label{sec: the sprawl}

If you don't intend for the game world to overlap with your sprawl game you can skip this section. \\

The game is designed to be compatible, with the sprawl, but really, your ultra-competent leatherclad assassin and Kimmy the Waif move in pretty different worlds. In essence, when you Hit the Streets, you are asking around the Peripherals. When you declare a contact, that contact might be a hot-shot VTOL pilot, but they might be Abrahim the Worm, a nobody in Hacker circles, but the guy to go to get a phone unlocked delicately. When your Killer powers down after a brutal rampage, someone has to hose the guts off and check the wires ain't worked loose. Some characters go home to sleep, others need a week in a ambrosia tank. Who's the doc?
Welcome to the Peripherals. 

There were two inspirations behind this game. I've spent enough time working in development engineering and living in places where a smartphone was easier to come by than a toilet to be interested in the lopsided tech of developing slums as reality chases past cyberpunk. Things like the favour-economy, where a cousin in the police makes an arrest record fall down the back of the cabinet but later on mentions his daughter would like to join your English school.  The same school being run by a friend who's an accountant by day, but you knew someone who could help get her (perfectly correct) paperwork signed off by a corrupt official. The other inspiration was the design space left by the Sprawl's excellent mission focused structure. It does what it does with a razor focus, but over time my players were interested in developing their characters non-business interactions with the world. There's also the matter of recurring NPCs in the sprawl game. It can be a fun change in pace 

So, the game. If you start with a Sprawl campaign, you've got a lot of the style choices, technologies, social problems ect already agreed. It makes sense for the same mega-corps to be present in the background, but makes much less sense for them to take a personal interest in the game. For that reason, the corporate clocks are not active by default. Stats are the same, and if a player wants to use an established NPC reasonable equipment that we've seen the NPC with should be included. If you want to play a Sprawl character in the periphery, use the "The Pro" playbook. You lose some moves because you're not on a mission, but you keep the stats, some moves, and some corporate interest. You get a hideaway, a mysterious reputation and the the ability to make worlds collide.

\chapter{Playbooks} \label{ch:playbooks}

\section{Breaker} \label{sec: Breaker}



\section{Goon} \label{sec:Goon}

\section{Trader} \label{sec:Trader}

\section{Prole} \label{sec:Prole}

\section{Outcast} \label{sec:Outcast}
GP

\section{Veteran} \label{sec:Veteran}
FS

\section{NGOwned} \label{sec:NGOwned}
CH

\section{Waif} \label{sec:Waif}
GP

\section{Bonesaw} \label{sec:Bonesaw}
SH


\section{The Pro} \label{sec:The Pro}
You lose some moves because you're not on a mission, but you keep the stats, some moves, and some corporate interest. You get a hideaway, a mysterious reputation and the the ability to make worlds collide.


\chapter{Cyberware} \label{ch:cyberware}
\section{Cyberware List} \label{sec: Cyberware list}

\chapter{Assets} 	\label{ch:assets}
\section {Favours}	\label{sec:favours}
\section {Cred} 	\label{sec:cred}
\section {Gangs}	\label{sec: gangs}
\section {Contacts}	\label{sec: contacts}

\chapter{Advancement}	\label{ch: advancement}
\chapter{The matrix}	\label{ch: the matrix}
\chapter{Running Cyber\_Peripherals}	\label{ch:running the game}
\section{agenda}			\label{sec: agenda}
\section{always say}		\label{sec: always say}
\section{principles}			\label{sec: principles}
\section{moves}			\label{sec: mcmoves}
\section{inflicting harm}		\label{sec:inflicting harm}
\section{countdown clocks}	\label{sec:countdown clocks}
\section{corporations} 		\label{sec: corporations}
\section{threats}			\label{sec: threats}
\section{putting it all together} 	\label{sec: putting it all together}

\chapter{the first session} \label{ch: the first session}

%% ---   ---   ---   ---   ---   
\chapter{Designer Notes}

Rule 1 - No Exceptions\\
Rule 2 - Needs to work for a one shot with strangers at a conference\\
Rule 3 - Evoke, not mimic, the sprawl and cyberpunk\\
Rule 4 - Support the GM\\
Rule 5 - Every mechanical part needs to justify its existence\\
Rule 6 - Every move needs to relate back to theme or the mechanics\\
Rule 7 - No stat substitution moves\\

on Playbooks\\
a) Does this playbook have a niche. Is it overshadowed?\\
b) Is the playbook fun and coherent?\\
c) Does the playbook offer levers for the GM to pull?\\
d) Does the playbook offer a second, different way to play that character?\\

On Moves\\
i) Is the trigger only going to trigger when it is important?\\
ii) does a miss move the sotry forward with something interesting?\\
iii) Does a success?\\
iv) \\

Formatting:\\
Only stats use digits. Hold is spent in one or twos not 1 or 2. Add explanation for hold or choice from list under the first option that uses it.\\
All moves to use the same format of trigger, 10+, 7-9 then 6-\\


%%
% The back matter contains appendices, bibliographies, indices, glossaries, etc.







\backmatter

\bibliography{sample-handout}
\bibliographystyle{plainnat}

\printindex[stuff]

\printindex

\end{document}

